\chapter{Discusión y conclusiones}

En el inicio de este documento se ha enunciado el propósito principal de presentar la regresión log-binomial como alternativa a los modelos de regresión logística en los estudios de cohorte y transversales debido a las limitaciones que se producen al utilizar este último: el \textit{odds ratio}, la medida de asociación de la que se sirve para medir la asociación entre la enfermedad y la exposición, tiene difícil interpretación y puede verse aumentado cuando se desea estimar el riesgo relativo o la razón de prevalencias en un estudio con resultados comunes en la población.\\

Es por este motivo por el que, siguiendo el planteamiento de \textcite{logbinom1}, se ha explicado detalladamente la regresión log-binomial: el porqué de su utilización, la formulación matemática, la estimación máximo verosímil de los parámetros, su interpretación mediante el riesgo relativo (o \textit{prevalence ratio} en el caso de estudios transverales) y el estudio de la bondad del ajuste del modelo. Además, se han expuesto varias formas de implementar el modelo en el \textit{software} estadístico R.\\

Sin embargo, al llevar a cabo el ajuste del modelo log-binomial computacionalmente, hay ocasiones en las que los algoritmos diseñados no consiguen encontrar el estimador máximo verosímil de los parámetros del modelo debido a que se dan problemas de convergencia.\\

Estas situaciones ocurren porque el modelo utiliza la función de enlace logarítmico y por lo tanto, para obtener una estimación de la probabilidad de enfermar ($Y$), basta con aplicar la función exponencial al predictor lineal (combinación de las variables explicativas y los parámetros a estimar), operación con la que se puede obtener cualquier valor mayor que 0. Esto conlleva la posibilidad que las probabilidas predichas de $Y$ estén fuera del intervalo $[0,1]$ y el consecuente problema de convergencia al maximizar la función de verosimilitud.\\

Para estos casos, se han presentado dos métodos alternativos. Por un lado, el uso de la función ``logbin" \text{del} paquete \textit{logbin} \autocite{logbinR} puesto que utiliza un algoritmo más estable con respecto a las propiedades de convergencia y por otro, el método ``COPY", propuesto por \textcite{COPY} e implementado en R por \textcite{Silvia}.\\

Ambos métodos, junto con la función ``glm", han sido aplicados a los datos de un estudio sobre la depresión postparto con el fin de ajustar diversos modelos de regresión log-binomial. También se han ajustado modelos logísticos para los mismos datos mediante ``glm" \text{y} se ha comprobado la bondad del ajuste de todos ellos.

\section{Comparación de resultados}

En el Capítulo \ref{cap:aplicacion} se ha podido comprobar, a partir de los resultados de los modelos dos y tres\footnote{Modelos para la incidencia de síntomas depresivos entre las 8 y las 32 semanas postparto y para la presencia de episodios de depresión mayor (DIGS) durante las 32 semanas postparto, respectivamente.}, que las estimaciones del \textit{odds ratio} y del riesgo relativo (en los modelos de regresión logística y log-binomial, respectivamente)  apuntaban en la misma dirección, es decir que indicaban lo mismo puesto que todos los riesgos relativos estimados ($\widehat{RR}$) han sido superiores a 1 y las respectivas estimaciones de los OR ajustados han tomado valores todavia más alejados de 1 ($\widehat{OR}_{adj} > \widehat{RR}$), como es obvio. \\

Asimismo, los parámetros estimados han resultado ser significativos en ambos modelos y sus signos han sido los mismos. Además, la prueba de \textit{Hosmer-Lemeshow} ha permitido determinar como buenos los ajustes de los modelos de ambas regresiones. \\

Por todo ello, parece ser que los modelo de regresión logística y log-binomial se podrían usar indistintamente. Aunque, como se puede observar a partir de los resultados del modelo uno\footnote{Modelo con la variable respuesta ``Presencia de síntomas depresivos a las 8 semanas postparto".}, hay casos en los que esta afirmación no es acertada.\\

Al comparar los resultados de ambas regresiones ajustadas para el último modelo mecionado (modelo 1), se han podido apreciar un cambio de signo en el valor de la estimación paramétrica asociada a la variable referente al apoyo social. También, un parámetro ha dejado de ser significativo en la regresión log-binomial respecto al modelo logístico (parámetro asociado a la variable de existencia de historial psiquiátrico personal) y parece ser que las estimaciones del \textit{odds ratio} y del riesgo relativo que se han obtenido llevan a diferentes conclusiones.  Por otra parte, en la evaluaión de la bondad del ajuste se ha obtenido que el modelo de regresión log-binomial (tanto el ajustado con ``glm'' como con ``logbin") no se ajusta bien a los datos.\\

En consecuencia, se concluye que la bondad del ajuste es de vital importancia para decidir el tipo de regresión a utilizar: se prefriere el modelo log-binomial por la interpretación del riesgo relativo, debido a que la entendemos mucho mejor, pero únicamente en esos casos en los que el ajuste es bueno. \\


Por otra parte, por lo que concierne al método ``COPY", en los tres modelos se ha escogido el ajuste en el que se han definido $n=10000$ copias a pesar de que los resultados obtenidos con $n=1000$ han sido muy similares. Esto ha sido debido a que el coste computacional en ambos casos es prácticamente el mismo por lo que, aunque la ganancia sea mínima, es preferible el ajuste con el mayor número de copias.

\section{Consideraciones metodológicas}

En el artículo ``\textit{A comparison of goodness-of-fit tests for the logistic regression model}" de \textcite{limitacioneshosmer} se explica que el test de bondad de ajuste de \textit{Hosmer-Lemeshow} tiene ciertas limitaciones: el valor del estadístico depende de la elección de los puntos de corte que definen los $g=10$ grupos y puede tener poca potencia para detectar ciertos tipos de falta de ajuste, es decir que su capacidad para rechazar la hipótesis nula de que el modelo se ajusta bien a los datos, incluso cuando esta es claramente falsa, es bastante débil. \\

Los autores sugieren, entre otras alternativas, la utilización del test propuesto por  \textcite{bondad}, el cual está disponible en R para el caso de los modelos de regresión logística a través de la función ``residuals.lrm". Como ya se ha mencionado en el Capítulo \ref{cap:estudioppal}, este es el método que fue utilizado en el estudio sobre la depresión postparto y es por ello que en este trabajo no se ha replicado. Además, el test no está implementado en R para el caso de los modelos de regresión log-binomial, por lo que no se hubiera podido realizar una comparación de ambos modelos, que es el objetivo principal del presente estudio. \\

Por último, mencionar que si se consigue implementar un buen algoritmo para ajustar un modelo log-binomial superando los problemas de convergencia y un test de bondad de ajuste que no posea las limitaciones comentadas en las pruebas actuales, la regresión log-binomial logrará ser una buena alternativa a la regresión logística como método para estimar el riesgo relativo en estudios de cohorte y transversales. Mientras tanto, en esos casos en los que se pueda ajustar el modelo log-binomial y en vista de los resultados obtenidos, es recomendable la utilización de la función ``glm''.