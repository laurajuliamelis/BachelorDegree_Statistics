%[formatcom=\color{Blue}]
\begin{Verbatim}

#####################################################################
########## ANÁLISIS DE DATOS ESTUDIO DEPRESIÓN POSTPARTO ###########
#####################################################################

# -------- 0. PAQUETES NECESARIOS. -------- #
#install.packages("epiDisplay")
library(epiDisplay)
#install.packages("logbin")
library(logbin)


# -------- 1. LECTURA DE DATOS. -------- #
load("DPP.RData")
head(dppdat)
labels(dppdat)
summary(dppdat)


# -------- 2. MODELOS DE REGRESIÓN LOGÍSTICA. -------- #
### 2.1. Síntomas de depresión a las 8 semanas (EPDS>9).
logistica8 <- glm(epds8wc~epqnT+epds0+duke+antpers, data=dppdat,
family=binomial(link="logit"))
summary(logistica8)
logistic.display(logistica8)

### 2.2. Síntomas de depresión a las 32 semanas (EPDS>9).
logistica32 <- glm(epds32winc~epqnT+epds0+ecoprob, data=dppdat,
family=binomial(link="logit"))
summary(logistica32)
logistic.display(logistica32)

### 2.3. Diagnóstico clínico de la depresión postparto (DIGS).
logisticaDIGS <- glm(digs0a32~epqnT+epds0+antpers+antpsifa+expvit, 
data=dppdat,family=binomial(link="logit"))
summary(logisticaDIGS)
logistic.display(logisticaDIGS)


# -------- 3. MODELOS DE REGRESIÓN LOG-BINOMIAL CON FUNCIÓN GLM. -------- #
### 3.1. Síntomas de depresión a las 8 semanas (EPDS>9).
logbinonial8 <- glm(epds8wc~epqnT+epds0+duke+antpers, data=dppdat,
 family=binomial(link="log"))
summary(logbinonial8)
exp(coef(logbinonial8))[-1]

##### Con argumento "start".
logbinonial8.start1 <- glm(epds8wc~epqnT+epds0+duke+antpers, data=dppdat, 
family=binomial(link="log"), start=c(-4, rep(0,4)))
summary(logbinonial8.start1)
exp(coef(logbinonial8.start1))[-1]


### 3.2. Síntomas de depresión a las 32 semanas (EPDS>9).
logbinomial32 <- glm(epds32winc~epqnT+epds0+ecoprob, data=dppdat, 
family=binomial(link="log"))
summary(logbinomial32)
exp(coef(logbinomial32))[-1]


### 3.3. Diagnóstico clínico de la depresión postparto (DIGS).
logbinomialDIGS <- glm(digs0a32~epqnT+epds0+antpers+antpsifa+expvit,
 data=dppdat, family=binomial(link="log"))
summary(logbinomialDIGS)
exp(coef(logbinomialDIGS))[-1]

##### Con argumento start.
logbinomialDIGS.start1 <- glm(digs0a32~epqnT+epds0+antpers+antpsifa+expvit, 
data=dppdat, family=binomial(link="log"), start=c(-4, rep(0,5)))
summary(logbinomialDIGS.start1)
exp(coef(logbinomialDIGS.start1))[-1]


# -------- 4. MODELOS DE REGRESIÓN LOG-BINOMIAL CON FUNCIÓN LOGBIN. -------- #
### 4.1. Síntomas de depresión a las 8 semanas (EPDS>9).
logbin8 <- logbin(epds8wc~epqnT+epds0+duke+antpers, data=dppdat)
summary(logbin8)
exp(coef(logbin8))[-1]

##### Con argumento bound.tol.
logbin8.bound <- logbin(epds8wc~epqnT+epds0+duke+antpers, data=dppdat, 
bound.tol= 0.05)
summary(logbin8.bound)
exp(coef(logbin8.bound))[-1]

### 4.2. Síntomas de depresión a las 32 semanas (EPDS>9).
logbin32 <- logbin(epds32winc~epqnT+epds0+ecoprob, data=dppdat)
summary(logbin32)
exp(coef(logbin32))[-1]

### 4.3. Diagnóstico clínico de la depresión postparto (DIGS).
logbinDIGS <- logbin(digs0a32~epqnT+epds0+antpers+antpsifa+expvit, data=dppdat)
summary(logbinDIGS)
exp(coef(logbinDIGS))[-1]

##### Con argumentos "bound.tol" y "maxit".
logbinDIGS.bound <- logbin(digs0a32~epqnT+epds0+antpers+antpsifa+expvit, 
data=dppdat, maxit=100, bound.tol= 0.05)
summary(logbinDIGS.bound)
exp(coef(logbinDIGS.bound))[-1]


# -------- 5. MODELOS DE REGRESIÓN LOG-BINOMIAL CON FUNCIÓN COPY. -------- #
### 5.1. Creación de la función COPY.
copy <- function(data, Y, vars,n, W) {
	if (!is.numeric(data[, Y])) {
		data[, Y] <- as.numeric(data[, Y])-1
	}
	data$W <- (n-1)/n
	data.copy <- data
	data.copy[, Y] <- 1-data.copy[, Y]
	data.copy$W <- 1/n
	data.all <- merge(data, data.copy, all = T)
	formul <- paste(Y, paste(vars, collapse = " + "), sep = "~")
	mod.mat <- model.matrix(as.formula(formul), data)
	glm.copy <- glm(as.formula(formul), family = binomial(log), data.all,
		    weights = W, control = list(maxit = 100),
		    start = c(-4, rep(0, ncol(mod.mat)-1)))
	return(glm.copy)
}

### 5.2. Síntomas de depresión a las 8 semanas (EPDS>9).
##### 5.2.1.Con n=100 copias.
logbinonial8.copy100 <- copy(dppdat,'epds8wc', vars = c('epqnT','epds0',
'duke','antpers'),100)
summary(logbinonial8.copy100)
exp(coef(logbinonial8.copy100))[-1]

##### 5.2.2.Con n=1000 copias.
logbinonial8.copy1000 <- copy(dppdat,'epds8wc', vars = c('epqnT','epds0',
'duke','antpers'),1000)
summary(logbinonial8.copy1000)
exp(coef(logbinonial8.copy1000))[-1]

##### 5.2.3.Con n=10000 copias.
logbinonial8.copy10000 <- copy(dppdat,'epds8wc', vars = c('epqnT','epds0',
'duke','antpers'),10000)
summary(logbinonial8.copy10000)
exp(coef(logbinonial8.copy10000))[-1]


### 5.3. Síntomas de depresión a las 32 semanas (EPDS>9).
##### 5.3.1.Con n=100 copias.
logbinonial32.copy100 <- copy(dppdat,'epds32winc', vars = c('epqnT','epds0',
'ecoprob'),100)
summary(logbinonial32.copy100)
exp(coef(logbinonial32.copy100))[-1]

##### 5.3.2.Con n=1000 copias.
logbinonial32.copy1000 <- copy(dppdat,'epds32winc', vars = c('epqnT','epds0',
'ecoprob'),1000)
summary(logbinonial32.copy1000)
exp(coef(logbinonial32.copy1000))[-1]

###### 5.3.3.Con n=10000 copias.
logbinonial32.copy10000 <- copy(dppdat,'epds32winc', vars = c('epqnT','epds0',
'ecoprob'),10000)
summary(logbinonial32.copy10000)
exp(coef(logbinonial32.copy10000))[-1]


### 5.4. Diagnóstico clínico de la depresión postparto (DIGS).
##### 5.4.1.Con n=100 copias.
logbinonialDIGS.copy100 <- copy(dppdat,'digs0a32', vars = c('epqnT','epds0',
'antpers','antpsifa','expvit'),100)
summary(logbinonialDIGS.copy100)
exp(coef(logbinonialDIGS.copy100))[-1]

##### 5.4.2.Con n=1000 copias.
logbinonialDIGS.copy1000 <- copy(dppdat,'digs0a32', vars = c('epqnT','epds0',
'antpers','antpsifa','expvit'),1000)
summary(logbinonialDIGS.copy1000)
exp(coef(logbinonialDIGS.copy1000))[-1]

###### 5.4.3.Con n=10000 copias.
logbinonialDIGS.copy10000 <- copy(dppdat,'digs0a32', vars = c('epqnT','epds0',
'antpers','antpsifa','expvit'),10000)
summary(logbinonialDIGS.copy10000)
exp(coef(logbinonialDIGS.copy10000))[-1]


# -------- 6. PRESENCIA DE SEPARACIÓN EN LOS MODELOS LOG-BINOMIALES 1 Y 3. -------- #
#install.packages("safeBinaryRegression")
library(safeBinaryRegression)

### 6.1. Síntomas de depresión a las 8 semanas (EPDS>9).
glm(epds8wc~epqnT+epds0+duke+antpers, data=dppdat, family=binomial(link="log"), 
separation ="find")

### 6.2. Diagnóstico clínico de la depresión postparto (DIGS).
glm(digs0a32~epqnT+epds0+antpers+antpsifa+expvit, data=dppdat, family=
binomial(link="log"), separation ="find")

# -------- 7. PRUEBAS DE BONDAD DE AJUSTE CON HOSMER-LEMESHOW. -------- #
### 7.1. Modelos de regresión logística. 
hoslem.test(logistica8$y, fitted(logistica8), g=10)
hoslem.test(logistica32$y, fitted(logistica32), g=10)
hoslem.test(logisticaDIGS$y, fitted(logisticaDIGS), g=10)

### 7.2. Modelos de regresión log-binomial.
###### 7.2.1.Con función "glm".
hoslem.test(logbinonial8.start1$y, fitted(logbinonial8.start1), g=10)
hoslem.test(logbinomial32$y, fitted(logbinomial32), g=10)
hoslem.test(logbinomialDIGS.start1$y, fitted(logbinomialDIGS.start1), g=10)

###### 7.2.1.Con función "logbin".
hoslem.test(logbin8$y, fitted(logbin8), g=10) 
hoslem.test(logbin32$y, fitted(logbin32), g=10)
hoslem.test(logbinDIGS$y, fitted(logbinDIGS), g=10)

###### 7.2.1.Con función "COPY".
hoslem.test(logbinonial8.copy100$y, fitted(logbinonial8.copy100), g=10)
hoslem.test(logbinonial8.copy1000$y, fitted(logbinonial8.copy1000), g=10)
hoslem.test(logbinonial8.copy10000$y, fitted(logbinonial8.copy10000), g=10)

hoslem.test(logbinonial32.copy100$y, fitted(logbinonial32.copy100), g=10)
hoslem.test(logbinonial32.copy1000$y, fitted(logbinonial32.copy1000), g=10)
hoslem.test(logbinonial32.copy10000$y, fitted(logbinonial32.copy10000), g=10)

hoslem.test(logbinonialDIGS.copy100$y, fitted(logbinonialDIGS.copy100), g=10)
hoslem.test(logbinonialDIGS.copy1000$y, fitted(logbinonialDIGS.copy1000), g=10)
hoslem.test(logbinonialDIGS.copy10000$y, fitted(logbinonialDIGS.copy10000), g=10)
\end{Verbatim}